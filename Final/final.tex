\documentclass[10pt]{article}
\usepackage{extsizes}
\usepackage{geometry}
\usepackage{amsmath}
\usepackage{setspace}
\usepackage{commath}
\onehalfspacing
\geometry{margin=0.5in}
\author{Dylan Garza}
\date
\maketitle
\begin{document}
\begin{titlepage}
\vspace*{\stretch{1.0}}
   \begin{center}
      \Large\textbf{CS final}\\
      \large\textit{  }
   \end{center}
   \vspace*{\stretch{2}}
\end{titlepage}
\section{Quiz 1}
\subsection*{question 1}
\textbf{This symbol marks the beginning of a comment in Python}
\begin{itemize}
\item \# $\leftarrow$
\item \&
\item **
\item *
\end{itemize}

\subsection*{question 2}
\textbf{A \_\_\_ is a name that represents a value that does not change during a program's execution}
\begin{itemize}
\item variable signature
\item named constant $\leftarrow$
\item key term
\item named literal
\end{itemize}
\subsection*{question 3}
\textbf{Which of the following statemens will cause an error?}
\begin{itemize}
\item x = '17'
\item 17 = x $\leftarrow$
\item x = 17
\item x = 9999
\end{itemize}
\subsection*{question 4}
\textbf{This operator performs integer divison:}
\begin{itemize}
\item \%
\item // $\leftarrow$
\item *
\item **
\end{itemize}
\subsection*{question 5}
\textbf{which is NOT a legal identifier?}
\begin{itemize}
\item 7thheaven $\leftarrow$
\item outrageouslyAndShockinglyLongRunon
\item lovePotionNumber9
\item \_42
\end{itemize}
\subsection*{question 6}
\textbf{Of the following variable names, which is the best one for keeping track of whether a patient has a fever or not?}
\begin{itemize}
\item feverTest
\item Temperature
\item fever
\item hasFever $\leftarrow$
\end{itemize}
\subsection*{question 7}
\textbf{the character escape sequence to force the cursor to go to the next line is:}
\begin{itemize}
\item $\backslash$n $\leftarrow$
\item $\backslash$t
\item $\backslash$'
\item $\backslash$"
\end{itemize}
\subsection*{question 8}
\textbf{which of the following is used to take the input in float?}
\begin{itemize}
\item float(input()) $\leftarrow$
\item float(float())
\item string(input())
\item int(input())
\end{itemize}
\subsection*{question 9}
\textbf{The following code multiplies 10 times the quantity 5 + 3, yielding the result 80: 10 * 5 + 3}
\begin{itemize}
\item True
\item False $\leftarrow$
\end{itemize}
\subsection*{question 10}
\textbf{the value of the expression '7' + '3' is the:}
\begin{itemize}
\item integer 73
\item integer 10
\item string '10'
\item string '73' $\leftarrow$
\end{itemize}
\subsection*{question 11}
\textbf{what does the int function attempt to do in the following code?}
\begin{verbatim}
value = input('Enter an integer: ')
value = int(value)
\end{verbatim}
\begin{itemize}
\item Convert a non-object to an object
\item Convert an integer to a string
\item convert a string to an integer $\leftarrow$
\item None of the above
\end{itemize}
\subsection*{quesion 12}
\textbf{which of the following is false?}
\begin{itemize}
\item \begin{verbatim}
if number1 == number2:
   print(number1, 'is equal to', number2)
\end{verbatim}
\item each if statement consists of the keyword if, the condition to test, and a (:) followed by an intended body
\item forgetting the colon (:) after the condition is a common syntax error
\item each body of if statement contains zero or more statements $\leftarrow$
\end{itemize}
\subsection*{question 13}
\textbf{Information in the main memory unit is \_. Its typically lost when the computers power is turned off}
\begin{itemize}
\item constant
\item sticky
\item volatile $\leftarrow$
\item persistaent
\end{itemize}
\subsection*{question 14}
\textbf{A(n) \_ is the smallest data item in a computer. It can have the value 0 or 1}
\begin{itemize}
\item byte
\item field
\item bit $\leftarrow$
\item record
\end{itemize}
\subsection*{What value is produed when Python evaluates the following expression}
\begin{verbatim}
5 * (12.7 - 4) / 2
\end{verbatim}
\begin{itemize}
\item 29.5
\item 21.75 $\leftarrow$
\item 21
\item None of the above
\end{itemize}
\break
\section*{Quiz 2}
\subsection*{question 1}
\textbf{A(n) \_ is the process of inspecting data that has been input to a program to make sure it is valid before it is used in computation}
\begin{itemize}
\item Correcting data
\item correcting input
\item data checking
\item Input validation $\leftarrow$
\end{itemize}
\subsection*{question 2}
\textbf{in Python, an infinite loop occurs when the computer accesses the wrong memory address}
\begin{itemize}
\item True
\item False $\leftarrow$
\end{itemize}
\subsection*{question 3}
\textbf{Which of teh following represents an example to calculate the sum of the numbers (accumulator)?}
\begin{itemize}
\item total + number = total
\item number +=number 
\item total = number
\item total += number $\leftarrow$
\end{itemize}
\subsection*{question 4}
\textbf{What is the format for the while clause in Python?}
\begin{itemize}
\item while condition
\item while condition: statement
\item while condition: $\leftarrow$
\item while condition statement
\end{itemize}
\subsection*{question 5}
\textbf{what is not an example of an augmented assignment operator?}
\begin{itemize}
\item <= $\leftarrow$
\item /=
\item *=
\item -=
\end{itemize}
\subsection*{question 6}
\textbf{IN python the variable in the for lause is reffered to as the \_ because it is the target of an assignment at the beginning of each loop iteration}
\begin{itemize}
\item for variable
\item count variable
\item loop variable
\item target variable $\leftarrow$
\end{itemize}
\subsection*{question 7}
\textbf{When will the following loop terminate?}
\begin{verbatim}
while leep_on_going != 999:
\end{verbatim}
\begin{itemize}
\item when keep\_on\_going refers to value less than 999
\item when keep\_on\_going refers to a value not equal to 999
\item when keep\_on\_going refers to a value greater than 999
\item when keep\_on\_going refers to a value equal to 999 $\leftarrow$
\end{itemize}
\subsection*{question 8}
\textbf{What are the values that the variable num contains through the iterations of the following for loop?}
\begin{verbatim}
for num in range(4)
\end{verbatim}
\begin{itemize}
\item 1,2,3,4 
\item 0,1,2,3 $\leftarrow$
\item 0,1,2,3,4
\item 1,2,3
\end{itemize}
\subsection*{question 9}
\textbf{In flowcharting, the decision structure and the repition structure both use the diamond symbol to represent the condition that is tested}
\begin{itemize}
\item True $\leftarrow$
\item False 
\end{itemize}
\subsection*{question 12}
\textbf{What are the values that the varibale contains through the iterations of the following for loop?}
\begin{verbatim}
for num in range(2,9,2)
\end{verbatim}
\begin{itemize}
\item 2,3,4,5,6,7,8,9
\item 1,3,5,7,9
\item 2,5,8
\item 2,4,6,8 $\leftarrow$
\end{itemize}
\subsection*{question 10}
\textbf{Both of the following for clauses would generate the same number of loop iterations}
\begin{verbatim}
for num in range(4):
for num in range(1,5)
\end{verbatim}
\begin{itemize}
\item True $\leftarrow$
\item False
\end{itemize}
\subsection*{question 11}
\textbf{What type of loop stricture repeats the code based on the value of the Boolean expression}
\begin{itemize}
\item Boolean-controlled loop
\item count-controlled loop
\item number-controlled loop
\item  condition-controlled loop $\leftarrow$
\end{itemize}
\subsection*{question 13}
\textbf{When using the \_ logical operatorm one or both of the subexpressions must be true for the compound expression to be true}
\begin{itemize}
\item or $\leftarrow$
\item not
\item maybe 
\item and
\end{itemize}
\subsection*{question 14}
\textbf{Which of the folliowing is the correct if clause to determine whether y is in range 10 through 50, inclusive}
\begin{itemize}
\item if $>=$ 10 ad $y<=$50; $\leftarrow$
\item if 10 $>$ y and y $<$50;
\item if y$>=$10 or y$<=$ 50;
\item if 10 $<$ y or y $>$ 50;
\end{itemize}
\subsection*{question 15}
\textbf{A(n) \_ is any piece of data that is passed into a function when the function is called}
\begin{itemize}
\item local varuable 
\item argument $\leftarrow$
\item global variable
\item parameter
\end{itemize}
\subsection*{question 16}
\textbf{What will display after the folliowing code is executed?}
\begin{verbatim}
def main():
   magic(5)
def magic(num):
   answer = num + 2 * 10
   print(answer)
main():
\end{verbatim}
\begin{itemize}
\item 100
\item 25 $\leftarrow$
\item Nothing
\item 70
\end{itemize}
\break
\section*{Exam 1}
\subsection*{question 1}
\textbf{Functions can be called from statemtnts in the body of a loop and loops can be calledfrom within the body of a function}
\begin{itemize}
\item True
\item False $\leftarrow$
\end{itemize}
\subsection*{question 2}
\textbf{What is the result of the following Boolean expression given that x=5,y=3, and z=8?}
\begin{verbatim}
x < y or z > x
\end{verbatim}
\begin{itemize}
\item 8
\item False
\item True $\leftarrow$
\item 5
\end{itemize}
\subsection*{question 4}
\textbf{Which of the following is the correct if clause to determine whether y is in range 10 through 50 inclusive}
\begin{itemize}
\item if 10 $<$ y or y $>$ 50
\item if y $>=$ 10 and y$<=$50 $\leftarrow$
\item if 10 $>$ y and y $<$ 50
\item if y$>=$10 or y$<=$50
\end{itemize}
\subsection*{question 3}
\textbf{see quiz 2 question 9}
\subsection*{question 5}
\textbf{Reducing duplication of code is one of the advantages of using loop structure}
\begin{itemize}
\item True $\leftarrow$
\item False
\end{itemize}
\subsection*{question 6}
\textbf{After the execution of the following statement, the variable price will reverence the value \_}
\begin{verbatim}
price = int(68,549)
\end{verbatim}
\begin{itemize}
\item 68 $\leftarrow$
\item 68.6
\item 68.55
\item 69
\end{itemize}
\subsection*{question 7}
\textbf{see quiz 2 question 12}
\subsection*{question 8}
\textbf{different functions can have local variables with the same names}
\begin{itemize}
\item True $\leftarrow$
\item False
\end{itemize}
\subsection*{question 9}
\textbf{The main reason to use secondary storage is to hold date for long periods of time even when the power supply to the computer is turned off}
\begin{itemize}
\item True $\leftarrow$
\item False 
\end{itemize}
\subsection*{question 10}
\textbf{The python language is not sensitive to block sturcturing of code.}
\begin{itemize}
\item True
\item False $\leftarrow$     
\end{itemize}
\subsection*{question 11}
\textbf{the not operator is a unary operator which must be used in a compound expression}
\begin{itemize}
\item True
\item False $\leftarrow$     
\end{itemize}
\subsection*{question 12}
\textbf{Whihc of the following is the correct if clause to determine wheter choice is anything other than 10?}
\begin{itemize}
\item if choice != 10
\item if choice <> 10
\item if not(choice < 10 and choice > 10):
\item if choice != 10: $\leftarrow$
\end{itemize}
\subsection*{question 13}
\textbf{the first line in a while loop is reffered to as the condition clause}
\begin{itemize}
\item True
\item False $\leftarrow$
\end{itemize}
\subsection*{question 14}
\textbf{A function defginition specifies what a function does and causes the function to execute}
\begin{itemize}
\item True
\item False $\leftarrow$
\end{itemize}
\subsection*{question 15}
\textbf{Both of the following for clauses would generate the same number of loop iterations.}
\begin{verbatim}
for num in range(4):
for num in range(1,5):
\end{verbatim}
\begin{itemize}
\item True $\leftarrow$
\item False
\end{itemize}
\subsection*{question 16}
\textbf{see quiz 2 question 5}
\subsection*{question 17}
\textbf{What does the following statement mean?}
\begin{verbatim}
num1, num2 = get_num()
\end{verbatim}
\begin{itemize}
\item the function get\_num()  will recieve the values stored in num1 and num2
\item the function get\_num() is expected to return one value and assign it ot num1 and num2
\item the statement will cause a syntax error
\item the function get\_num() is expected to return value for num1 and for num2 $\leftarrow$
\end{itemize}
\subsection*{question 18}
\textbf{what does the following program do?}
\begin{verbatim}
student = 1
while student <=3:
   total = 0
   for score in range(1,4):
      score = int(input("enter test score:"))
      total += score
   average = total/3
   print("Student",student,"average:",average)
   student += 1
\end{verbatim}
\begin{itemize}
\item It accepts 3 test scores for each of 3 students and outputs the average for each student $\leftarrow$
\item It accepts one test score for each of 3 students and outputs teh average of the 12 scores
\item it accepts 4 test scores for 3 students and outputs the average of the 12 scores
\item it accepts 4 test scores 2 students, then averages and outputs all the scores
\end{itemize}
\subsection*{question 19}
\textbf{\_ is the process of inspecting data that has been input into a program in order to ensure that the data is valid before it is used in a computation}
\begin{itemize}
\item Data validation
\item Correcting input
\item Correcting data
\item input validtion $\leftarrow$
\end{itemize}
\subsection*{question 20}
\textbf{THe python language uses a compiler which is a program that both translates and executes the instruction in a high-level language}
\begin{itemize}
\item True
\item False $\leftarrow$
\end{itemize}
\subsection*{question 21}
\textbf{Which tpye of error preventsthe program from running?}
\begin{itemize}
\item logical 
\item grammatical
\item human
\item syntax $\leftarrow$
\end{itemize}
\subsection*{question 22}
\textbf{according to the behavior of integer division is divided by an integer, the result will be a float}
\begin{itemize}
\item True 
\item False $\leftarrow$
\end{itemize}
\subsection*{question 23}
\textbf{In python, print statements written on sperate lines do not necessarily ouput on seperate lines}
\begin{itemize}
\item True $\leftarrow$
\item False
\end{itemize}
\subsection*{question 24}
\textbf{since a named constant is just a variable, it can change any time during a program's execution}
\begin{itemize}
\item True
\item False $\leftarrow$
\end{itemize}
\subsection*{question 25}
\textbf{what is the informal language used by programmers use to create models of programs that has no syntax rules and is not meant to be compiled or executed?}
\begin{itemize}
\item source code 
\item flowchart
\item pseudocode $\leftarrow$
\item algorithm
\end{itemize}
\subsection*{question 27}
\textbf{which of the following will display 20\%?}
\begin{itemize}
\item \begin{verbatim} print(format(0.2,'%')) <enter> \end{verbatim}
\item \begin{verbatim} print(format(20,'.0%')) <enter> \end{verbatim}
\item \begin{verbatim} print(format(0.2 * 100, ':0%')) <enter> \end{verbatim}
\item $\rightarrow$\begin{verbatim} print(format(0.2, ':0%')) <enter> \end{verbatim} 
\end{itemize}
\subsection*{question 26}
\textbf{after the execution of the following statement thevariable sold will reference the numeric literal value as (n) \_ data type}
\begin{verbatim}
sold = 256.752
\end{verbatim}
\begin{itemize}
\item currency
\item int
\item float $\leftarrow$
\item str
\end{itemize}
\subsection*{question 28}
\textbf{the if statement causes one or more statemnts to execute only when a Boolean expression is true}
\begin{itemize}
\item True $\leftarrow$
\item False
\end{itemize}
\subsection*{question 29}
\textbf{Python allows you to compare strings, but it is not case sensitive}
\begin{itemize}
\item True
\item False $\leftarrow$
\end{itemize}
\subsection*{question 30}
\textbf{A(n) \_ strucure is a logical design that controls the order in which a set of statements execute}
\begin{itemize}
\item function
\item iteration
\item sequence
\item control $\leftarrow$
\end{itemize}
\subsection*{question 31}
\textbf{When using the \_\_ logical operator both submissions must be true for the compound expression to be true}
\begin{itemize}
\item and $\leftarrow$ 
\item either or and
\item not 
\item or
\end{itemize}
\subsection*{question 32}
\textbf{See quiz 2 question 2}
\subsection*{question 33}
\textbf{When will the following loop terminate?}
\begin{verbatim}
while keep_on_going != 999:
\end{verbatim}
\begin{itemize}
\item when it refers to a value equal to 999 $\leftarrow$   
\item when it refers to a value not equal 999
\item when it refers to a value greater than 999
\item when it refers to a value less than 999 
\end{itemize}
\subsection*{question 34}
\textbf{A hierarchy chart shows all the steps that are taken inside a function}
\begin{itemize}
\item True
\item False $\leftarrow$
\end{itemize}
\subsection*{question 35}
\textbf{It is recommended  that programmera avoid \_ varuables in a program whenever possible}
\begin{itemize}
\item string
\item local
\item global $\leftarrow$
\item keyword
\end{itemize}
\subsection*{question 36}
\textbf{Which of the following will assign a random integer in the range of 1 through 50 to variable a number?}
\begin{itemize}
\item number = random.rantin(1,50) $\leftarrow$
\item number = random(range(1,50)) 
\item random(1,50) = number
\item randint(1,50) = number
\end{itemize}
\subsection*{question 37}
\textbf{what will the output after the following code is executed}
\begin{verbatim}
def pass_it(x,y):
   z = x,",",y
num1 = 4
num2 = 8
answer = pass_it(num1,num2)
print(answer)
\end{verbatim}
\begin{itemize}
\item 4,8
\item None $\leftarrow$
\item 8,4
\item 48
\end{itemize}
\subsection*{question 38}
\textbf{A software developer is the person with teh training to design, create, and test computer programs}
\begin{itemize}
\item True $\leftarrow$
\item False
\end{itemize}
\subsection*{question 39}
\textbf{Which of the following returns the largest integer that is less than or equal to its argument}
\begin{itemize}
\item greater
\item lesser
\item ceil
\item floor $\leftarrow$
\end{itemize}
\subsection*{question 41}
\textbf{The process known as the \_ cycle is used by the CPU to execute instructions in a program}
\begin{itemize}
\item decode-fetch-execute 
\item fetch-decode-execute
\item decode-execute-fetch
\item fetch-execute-decode
\end{itemize}
\subsection*{question 40}
\textbf{In a nested loop, the inner loop goes through all of its iterations foe each for each iterations of the outer loop}
\begin{itemize}
\item True $\leftarrow$
\item False
\end{itemize}
\subsection*{question 42}
\textbf{In Python, math expressions are always evaluated left ot right no matter what the operators are}
\begin{itemize}
\item True
\item False $\leftarrow$
\end{itemize}
\subsection*{question 43}
\textbf{the \_\_\_  function reads a piece of data that has been entered at the keybpoard and returns that piece of data as a string back to the program}
\begin{itemize}
\item eval\_input()
\item str\_input()
\item input() $\leftarrow$
\item output()
\end{itemize}
\subsection*{question 44}
\textbf{the randrange function returns a randomly selected value from a specific sequence of numbers}
\begin{itemize}
\item True $\leftarrow$ 
\item False
\end{itemize}
\subsection*{question 45}
\textbf{Python formats all floating-point numbers to two decimal places when outputting with the print statement}
\begin{itemize}
\item True
\item False $\leftarrow$
\end{itemize}
\subsection*{questions 46}
\textbf{short-circuit evaluation is only performed withthe not operator}
\begin{itemize}
\item True
\item False $\leftarrow$
\end{itemize}
\subsection*{questions 47}
\textbf{what is the result of the following Boolean expression, given that}
\begin{verbatim}
x = 5, y = 3, z = 8
not (x < y or z > x) and y < z
\end{verbatim}
\begin{itemize}
\item 8 
\item True
\item 5
\item False $\leftarrow$
\end{itemize}
\subsection*{question 48}
\textbf{python uses the same symbols for the assignment operator as for the equality operator}
\begin{itemize}
\item True
\item false $\leftarrow$
\end{itemize}
\subsection*{question 49}
\textbf{what type of loop structure repeats the code a specific number of times?}
\begin{itemize}
\item Boolean-controlled loop
\item count-controlled loop $\leftarrow$
\item number-controlled loop
\item condition-controlled loop
\end{itemize}
\subsection*{question 50}
\textbf{Python function names follow the same rules as those for those for naming variables}
\begin{itemize}
\item True $\leftarrow$
\item False
\end{itemize}
\subsection*{question 51}
\textbf{you are working on a program that consists of one globa variable and multiple functions. any function in the program can access teh global variable}
\begin{itemize}
\item True $\leftarrow$
\item False
\end{itemize}
\subsection*{question 52}
\textbf{if a sentinel value is not distinct the chosen value may appear as a value in the programs normla execution. this will result in early and undesired termination of the program}
\begin{itemize}
\item True $\leftarrow$
\item False
\end{itemize}
\subsection*{question 53}
\textbf{A Boolean expression is an expression that produces either int or float numbers}
\begin{itemize}
\item True
\item false $\leftarrow$
\end{itemize}
\subsection*{question 54}
\textbf{ifa a math expression adds a float to an int, the data type of the result will be int}
\begin{itemize}
\item True
\item False $\leftarrow$
\end{itemize}
\subsection*{Question 55}
\textbf{True, False, and None are keywards in Python}
\begin{itemize}
\item True $\leftarrow$
\item False
\end{itemize}
\break
\section*{Quiz 3}
\subsection*{question 1}
\textbf{When Working with a sequential access file, you can jump directly to any piece of data in the file without reading the data that comes before it}
\begin{itemize}
\item True
\item False $\leftarrow$ 
\end{itemize}
\subsection*{question 2}
\textbf{When you open a file that file already exists on teh disk using the "w" mode, the contents of the existing file will be erased}
\begin{itemize}
\item True $\leftarrow$
\item False
\end{itemize}
\subsection*{question 3}
\textbf{When a file that already exists is opened in append mode, the file's existing contents are erased}
\begin{itemize}
\item True
\item False $\leftarrow$
\end{itemize}
\subsection*{question 4}
\textbf{If you do not handle an exception, it is ignored by the Python interpretor and the program continues to execute}
\begin{itemize}
\item True
\item False $\leftarrow$
\end{itemize}
\subsection*{question 5}
\textbf{You can have more than one except clause in a try/except statement}
\begin{itemize}
\item True $\leftarrow$
\item False
\end{itemize}
\subsection*{question 6}
\textbf{The finally suite in a try/except statement executes only if no exceptions are raised by statements in the try suite}
\begin{itemize}
\item True
\item False $\leftarrow$
\end{itemize}
\subsection*{question 7}
\textbf{When a program is finished using a file, it should do this}
\begin{itemize}
\item encrypt the file
\item erase the file
\item open the file
\item close the file $\leftarrow$
\end{itemize}
\subsection*{question 8}
\textbf{The following code will display 'yes' + 'no'}
\begin{verbatim}
mystr = 'yes'
yourstr = 'no'
mystr += yourstr
print(mystr)
\end{verbatim}
\begin{itemize}
\item True
\item False $\leftarrow$
\end{itemize}
\subsection*{question 9}
\textbf{What will be assigned to the variable s\_string after the following code executes?}
\begin{verbatim}
special = '1357 Country Ln.'
s_string = special[:4]
\end{verbatim}
\begin{itemize}
\item '1357' $\leftarrow$
\item '7'
\item 5
\item '7 Country Ln.'
\end{itemize}
\subsection{question 10}
\textbf{Indexing of a string starts at 1 so the index of the first character is 1, the index of the second character is 2 and so forth}
\begin{itemize}
\item True
\item False $\leftarrow$
\end{itemize}
\subsection*{question 11}
\textbf{What will be the value of the variable string after the following code executes?}\begin{verbatim}
string = 'abcd'
string.upper()
\end{verbatim}
\begin{itemize}
\item 'Abcd'
\item Nothing: this code is invalid
\item 'ABCD' $\leftarrow$
\item 'abcd'
\end{itemize}
\subsection*{question 12}
\textbf{The index -1 indentifies the last element in a list}
\begin{itemize}
\item True $\leftarrow$
\item False
\end{itemize}
\subsection*{quesion 13}
\textbf{Which list will be referemced by the variable number after the following code is executed}
\begin{verbatim}
number = range(0,9,2)
\end{verbatim}
\begin{itemize}
\item $[$0,1,2,3,4,5,6,7,8,9$]$
\item $[$1,3,5,6,9$]$
\item $[$2,3,6,8$]$
\item $[$0,2,4,6,8$]$ $\leftarrow$
\end{itemize}
\subsection*{quesion 14}
\textbf{What will be the value of the variable list after the following code executes?}
\begin{verbatim}
list = [1,2,3,4]
list[3] = 10
\end{verbatim}
\begin{itemize}
\item Nothing: this code is invalid
\item $[$1,2,3,10$]$ $\leftarrow$
\item $[$1,2,10,4$]$
\item $[$1,10,10,10$]$
\end{itemize}
\subsection*{question 15}
\textbf{List are mutable, which mean their elements can be changed in a program}
\begin{itemize}
\item True$\leftarrow$
\item False
\end{itemize}
\break
\section*{Quiz 5}
\subsection*{question 1}
\textbf{OOP allows us to hide object data atributes from code that is outside the object}
\begin{itemize}
\item True $\leftarrow$
\item False
\end{itemize}
\subsection*{question 2}
\textbf{The instances of a class share data attributes in the class}
\begin{itemize}
\item True
\item False $\leftarrow$
\end{itemize}
\subsection*{question 3}
\textbf{The self parameter need not be named self, but it is strongly reccomended to conform with standard practice}
\begin{itemize}
\item True $\leftarrow$
\item False
\end{itemize}
\subsection*{question 4}
\textbf{what type of programming contains class definitions?}
\begin{itemize}
\item Procedural
\item Object
\item Object-oriented $\leftarrow$
\item Modular
\end{itemize}
\subsection*{question 5}
\textbf{what are the procedures that an object performs called?}
\begin{itemize}
\item Methods $\leftarrow$
\item Actions
\item Instances
\item Modules
\end{itemize}
\subsection*{question 6}
\textbf{What is the combining of data and code in a single object known as?}
\begin{itemize}
\item{Objectification}
\item Encapsulation $\leftarrow$
\end{itemize}
\subsection*{question 7}
\textbf{A(n) is a component of a class that references data}
\begin{itemize}
\item data attribute $\leftarrow$
\end{itemize}
\subsection*{question 8}
\textbf{by doing this you can hide a class's attribute from code outside the class.}
\begin{itemize}
\item begin the name of the attribute with private\_\_
\item begin the name of the attribute with the @ symbol
\item begin the attribute's name with two underscores $\leftarrow$
\item avoid using the self parameter to create attributes
\end{itemize}
\subsection*{question 9}
\textbf{what is the special name given to the method that returns a string containing the object's state?}
\begin{itemize}
   \item \_\_state\_\_      
\item \_\_obj\_\_
\item \_\_str\_\_ $\leftarrow$
\item \_\_init\_\_
\end{itemize}
\subsection*{question 10}
\textbf{which method is automatically executed when an instance of the class is created in memory}
\begin{itemize}
\item \_\_init\_\_ $\leftarrow$
\end{itemize}
\subsection*{question 11 }
\textbf{The differnece of set1 and set2 is a set that contrains only the elements that appear in set1 but do not appear in set2}
\begin{itemize}
\item True $\leftarrow$
\item False
\end{itemize}
\subsection*{question 12}
\textbf{The elements in a dictionary are stored in ascening order,by the keys of the key-valye pairs}
\begin{itemize}
\item True
\item False $\leftarrow$
\end{itemize}
\subsection*{question 13}
\textbf{the issubset() methid can be used to determine whether set1 is a subset of set2}
\begin{itemize}
\item True $\leftarrow$
\item False
\end{itemize}
\subsection*{question 14}
\textbf{A dictionary can include the same value several times but cannot include the same key several times}
\begin{itemize}
\item True $\leftarrow$
\item False
\end{itemize}
\subsection*{question 15}
\textbf{what is the correct structure to create a dictionary of months where each month will be accessed by its month number, for example January is month 1 April is month 4?}
\begin{itemize}
\item 1: 'January'; 2: 'February' $\leftarrow$
\end{itemize}
\subsection*{question 16}
\textbf{What will be the result of the following code?}
\begin{verbatim}
ages = {'Aaron':6;'Kelly':3;'Abigail':1}
value = ages['Briana']
\end{verbatim}
\begin{itemize}
\item KeyError $\leftarrow$
\end{itemize}
\subsection*{question 17}
\textbf{What will be displayed after the code executes}
\begin{verbatim}
cities = ['GA':'Atlanta';'NY':'Albant';'CA':'San Diego']
if 'CA' in cities:
   del cities['CA']
   cities['CA'] = 'Sacramento'
print(cities)
\end{verbatim}
\begin{itemize}
\item \begin{verbatim}
{'CA':'Sacramento';'NY':'Albany';'GA':'Atlanta'}
   \end{verbatim} $\leftarrow$
\end{itemize}
\subsection*{question 18}
\textbf{What does the following print?}
\begin{verbatim}
D = {'a':2;'x':7;'d':5}
for k,v in D.items():
   print(k,v, end =',')
\end{verbatim}
\begin{itemize}
\item a 2,x 7,d 5,
\end{itemize}



























\end{document}
